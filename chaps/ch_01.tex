\chapter{Why We Count}
\noindent

When we begin to study mathematics at a very young age, we are taught to count using a set of numbers called the \textit{Natural Numbers}. We often start at 0 (although it is debated whether or not to count 0 as a natural number), which is representative of nothingness. It is a starting point, a way to quantify the beginning of an event, or a measure of nothingness. Some cultures saw 0 as something divine while others did all they could to avoid it, seeing it as a bad omen. We then move on to 1, from which we can generate all other numbers in $\NN$.

\section{Prehistoric Numbers}
\noindent

The Natural Numbers are a luxury. We can describe infinite discreet quantities using these numbers. This was not always the case, however. Some of the earliest cultures did not have a way (or a need) to describe any amount more than \textit{many}. The definition of \textit{many} differs from culture to culture, but it seems to lie in the neighborhood of greater than 6. The ancient tribe of the Gumulgal, located in Australia, appear to describe numbers up to 6 using only 1 and 2:
\begin{center}
    \begin{tabular}{|c | c | c |}
        \hline
        1 & 2 & 3 \\
        \hline
        \textit{urapon} & \textit{ukusar} & \textit{ukusar-urapon}\\
        \hline
        4 & 5 & 6\\
        \hline
        \textit{ukusar-ukusar} & \textit{ukusar-ukusar-urapon} & \textit{ukusar-ukusar-ukusar}\\
        \hline
    \end{tabular}
\end{center}
This form of counting was not isolated to the Gumulgal. Other tribes used symbols to identify numbers in a pseudo-base-two format:
\begin{center}
    \begin{tabular}{l | l | l}
        6 & 7 & 8\\
        \hline
        $|||$ & $||||$ & $||||$\\
        $|||$ & $|||$ & $||||$
    \end{tabular}
\end{center}
Such notation leads one to identify the first \textit{operation}: addition. We use the symbol $+$ to note the \textit{sum} of at least two numbers. The Gumulgal, for example would express addition by combining words:
\[
urapon + ukusar = ukusar\text{-}urapon,
\]
which roughly translates to
\[
1 + 2 = 3.
\]
More interesting still is the fact that other prehistoric tribes chose to express their numbers based off of something other than 1 or 2. The Yasayama from the Congo prefer to use 5 as a jumping off point. The notion of 1 is not lost on them, however, as they must fist reach 5. As is turns out, they have distinct words for the various numbers leading up to 5 (i.e. 1, 2, 3, and 4), but 6 is expressed as $5+1$, 7 is $5+2$, 8 is $5+3$, and so on until 10 which has a unique name. As it turns out, these tribes have a lot more to contribute to mathematics than simple counting systems.

We eventually learn, in our schooling, another operation that we use to express repeated addition. This operation, often referred to as multiplication, allows one to compute larger numbers without going through the tedium of adding repeatedly. We see this operation noted as $\times$, $\cdot$, $*$, or, if we are dealing with two variables, $xy$, where the symbol is left out and the implied operation is multiplication.

\section{Special Numbers}
