\chapter{Proofs and Set Theory}

\section{Logic}

We will begin by introducing the ideas of sets and proof simultaneously. Logic is the fundamental basis of all of mathematics; sets are the building blocks of all the structures that we know and love. There are various ways to prove the assertions in mathematics, but we will start by laying out basic logic rules and definitions. Much of the content of this section is a simplification of the actual logic we will use when proving important mathematical results; in other words, we won't see many truth tables in other chapters and sections. Nonetheless, the topics are fundamental and help us to put the rest of the content of this book into perspective. We will begin with a few definitions. This section will be a little heavy on the definitions as well as notation, but then again, all of mathematics is.

\begin{definition}{\deft{Statement}}
	A statement is an assertion that is either true or false.
\end{definition}

This may seem like a useless definition, but it actually points out a key fact that we will rely on for the rest of this book; that is, assertions are either true or false. Interestingly, there is a branch of mathematics that deals with statements that can be neither true nor false (often called \textit{multilogic}), but the ideas perpetuated by that field of study exceed the level of material covered here. The state of a statement is known as its truth value.

\begin{definition}{\deft{Truth Value}}
	The truth value of a statement is either true or false.
\end{definition}

Again, this definition is probably intuitive, but ultimately necessary. We typically name a statement $P$. With this in mind, we can begin to look at discrete logical statements.

\begin{definition}{\deft{Negation}}
	The negation of a statement $P$, denoted $\neg P$, is the statement that has the opposite truth value of $P$.
\end{definition}
\begin{definition}{\deft{Conjunction}}
	Let $P$ and $Q$ be statements. The conjunction of statements $P$ and $Q$ is the truth value of $P$ and $Q$, denoted $P \land Q$.
\end{definition}
\begin{definition}{\deft{Disjunction}}
	Let $P$ and $Q$ be statements. The disjunction of statements $P$ and $Q$ is the truth value of either $P$ or $Q$, or $P$ and $Q$, denoted $P \lor Q$.
\end{definition}
\begin{definition}{\deft{Implication}}
	Let $P$ and $Q$ be statements. The implication of statements $P$ and $Q$ is the truth value of $P$ implying $Q$, denoted $P\to Q$.
\end{definition}

These definitions are rather vague; the main purpose of pointing them out is to allow us to become familiar with the notation. Of more importance is the truth table created by the statements. A truth table is a handy way to determine the truth value of the four definitions given the truth values of $P$ and $Q$. The following is a truth table for the negation, conjunction, disjunction and implication of statements $P$ and $Q$. We will follow the convention set by most other mathematics textbooks: $T$ means \textit{true} and $F$ means \textit{false}.

\begin{center}
	\begin{tabular}{c | c | c | c | c | c | c}
		$P$ & $Q$ & $\neg P$ & $\neg Q$ & $P \land Q$ & $P \lor Q$ & $P \to Q$ \\
		\hline
		$T$ & $T$ & $F$      & $F$      & $T$         & $T$        & $T$       \\
		$T$ & $F$ & $F$      & $T$      & $F$         & $T$        & $F$       \\
		$F$ & $T$ & $T$      & $F$      & $F$         & $T$        & $T$       \\
		$F$ & $F$ & $T$      & $T$      & $F$         & $F$        & $T$       \\
	\end{tabular}
\end{center}

The truth table above is very useful and probably worth memorizing. The only caveat is the implication. The truth values for $P\to Q$ makes sense, however; if we assert that something is true, it cannot make sense to say that whatever follows is false. In other words statements like \textit{If Alicia is mortal, then she is a dog} do not make sense logically.

At this point, you may be wondering what happens if $P = Q$. We call this

\begin{definition}{\deft{Logical Equivalence}}
	Let $P$ and $Q$ be logical statements. We say $P$ is logically equivalent to $Q$ if the truth values of $P$ and $Q$ are the same. We use the notation $P\equiv Q$.
\end{definition}

\noindent We now have all we need to perform our first true proof. It turns out that truth tables can be used to show logical equivalence of two statements. Consider the following example.

\begin{example}{\egft{Determine if $\neg(P \land Q) \equiv (\neg P \lor \neg Q)$}}\label{eg:demorgan}
	We can use a truth table to prove this result. We will do this by list all possible combinations of $P$ and $Q$, and then determine the truth values of the statements. If the truth values of $\neg(P \land Q)$ and $(\neg P \lor \neg Q)$ are equivalent, we will have shown that the statements are logically equivalent.
	\begin{center}
		\begin{tabular}{c | c | c | c}
			$P$ & $Q$ & $\neg(P \land Q)$ & $(\neg P \lor \neg Q)$ \\
			\hline
			$T$ & $T$ & $F$               & $F$                    \\
			$T$ & $F$ & $T$               & $T$                    \\
			$F$ & $T$ & $T$               & $T$                    \\
			$F$ & $F$ & $T$               & $T$
		\end{tabular}
	\end{center}
	Thus, we conclude that $\neg(P \land Q) \equiv (\neg P \lor \neg Q)$.
\end{example}

The result of \cref{eg:demorgan} is actually a very important result known as DeMorgan's Law. The actual statement of the law is our first theorem.

\begin{namedtheorem}[DeMorgan's Law]\label{thm:demorgan}
	Let $P$ and $Q$ be statements. Then
	\[\neg(P \land Q) \equiv (\neg P \lor \neg Q),\] and
	\[\neg(P\lor Q)\equiv (\neg P \land \neg Q).\]
\end{namedtheorem}

DeMorgan's law is a very powerful mathematical theorem (we will define what a theorem is later) that we will use when proving results about sets. We won't prove the second statement in DeMorgan's law here (see problem \ref{prob:demorgan}), so you can take it for granted here.

Before we finish this section, we will introduce a few more logical operators and their truth tables. The first operator will look similar to disjunction, but with a small change (see problem \ref{prob:xor}).

\begin{definition}{\deft{Exclusive Disjunction}}
	The exclusive disjunction of logical statements $P$ and $Q$ is the disjunction of $P$ and $Q$ where if $P$ and $Q$ are true, then the exclusive disjunction is false. In other words,
	\begin{center}
		\begin{tabular}{c | c | c}
			$P$ & $Q$ & $P\oplus Q$ \\
			\hline
			$T$ & $T$ & $F$         \\
			$T$ & $F$ & $T$         \\
			$F$ & $T$ & $T$         \\
			$F$ & $F$ & $F$
		\end{tabular}
	\end{center}
\end{definition}

The exclusive disjunction, often referred to as \textit{exclusive or} or \textit{xor}, is not as useful as the next logical operator.

\begin{definition}{\deft{Biconditional Implication}}
	Let $P$ and $Q$ be statements. Then the biconditional implication of $P$ and $Q$ is the statement $(P\to Q)\land(Q\to P)$, denoted $P\iff Q$.
\end{definition}

The biconditional is often referred to by \textit{if and only if}, or, since mathematicians don't like to write a lot, \textit{iff}. This is a very powerful tool since it allows us to evaluate implication in both directions; it is also a slight annoyance, since we will have to prove both direction of statements that contain iff.

\subsection*{Problems}
\begin{enumerate}
	\item Evaluate the truth table for $(P\lor Q)\land\neg(P\land Q)$. What is this statement equivalent to?\label{prob:xor}
	\item Determine the truth table for $P \iff Q$, based on the definition using implication and conjunction.
	\item Determine if the following logical statements are equivalent.
	      \begin{enumerate}
	      	\item $(P\to Q)$ and $(P\land Q) \lor \neg(P\to Q)$.
	      	      %TODO -- come up with more
	      \end{enumerate}
	\item Prove the second logical equivalence in DeMorgan's law.\label{prob:demorgan}
	\item If we have 3 logical statements, how many rows will we have in a given truth table? What about 4? $n$?
\end{enumerate}

\newpage

\section{Proofs}

We will now begin to look at basic types of proof techniques. 

\newpage

\section{Sets}
