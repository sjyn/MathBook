\section{Normality, Cosets, and Factor Groups}

The topics covered in this section are arguably the most fundamental topics that we will cover with regards to groups. The ideas presented hear lay the foundation for some of the most important topics in group theory and beyond, so it is very important that we discuss the topics in depth.
We want to form an operation on a group that is well defined and ultimately forms what we call a factor group, but to do this we need to discuss two very important concepts. We will begin with the concept of a normal subgroup. We have seen normal subgroups before without ever calling them by name.

\subsection*{Normality}
Normality is a special condition that we place on a subgroup provided that it meets certain conditions. The first requirement is that the subset we are looking at is actually a subgroup, but we know how to prove that. The second condition, the condition that defines normality, requires something called the conjugate.
\begin{definition}{Conjugate}
	Let $G$ be a group and let $a\in G$. The conjugate of $a$ by $g$ is the expression
	\[
		gag^{-1}
	\]
	for some $g\in G$.
\end{definition}
We say that a given subgroup is normal provided that it is closed under conjugacy.
\begin{definition}{Normal Subgroup}
	Let $G$ be a group and let $H\leq G$. Then $H$ is normal in $G$, denoted $H\normal G$, if
	\[
		ghg^{-1}\in H\ \forall g\in G\land\ \forall h\in H.
	\]
\end{definition}

Normality allows us to mimic the behavior of abelian groups at a cost. For example, if we let $H\normal G$, then
\[
	ghg^{-1}=h'\implies gh=h'g^{-1}.
\]
In other words, we can commute $g$ by changing $h$. Generally, $h\neq h'$; we will see this in the following example.

\begin{example}{Show $SL_{n}(\RR)\normal GL_{n}(\RR)$}
	We have show that $SL_{n}(\RR)$ is a subgroup of $GL_{n}(\RR)$ already, so we just need the normality condition to be true. Let $A\in SL_{n}(\RR)$ and $X\in GL_{n}(\RR)$. Then
	\begin{align*}
		\det(XAX^{-1}) & =\det(X)\det(A)\det(X^{-1}) \\
		               & =\det(X)\det(X^{-1})        \\
		               & =\det(X)\det(X)^{-1}        \\
		               & =1.
	\end{align*}
	Therefore, we can conclude that $SL_{n}(\RR)\normal GL_{n}(\RR)$. Note that matrix multiplication does not always commute, so we have that
	\[
		XAX^{-1}\in SL_{n}(\RR)\not\implies XA=AX.
	\]
\end{example}

The general linear group is an example of a non-abelian group. It turns out that every subgroup of an abelian group is normal since we can say $ghg^{-1}=gg^{-1}h=h$, and it follows that the center of any group is normal to that group.
\begin{theorem}{}
	Let $G$ be a group. Then $Z(G)\normal G$.
\end{theorem}
\begin{proof}
	Let $z\in Z(G)$. Then $gzg^{-1}=gg^{-1}z=z\in Z(G)$.
\end{proof}

We do not, however, require that the group be abelian in order to have every subgroup be normal. One example of a non-abelian group in which every subgroup is normal is the $Q_{8}$, called the quaternion group. This group is defined in the following way:
\[
	Q_{8}={\pm i, \pm j,\pm k, \pm 1}
\]
under multiplication. The following Cayley table shows the multiplication of the positive elements; if we want the negative, we can multiply by $-1$.
\begin{center}
	\begin{tabular}{c | c c c c}
		    & $1$ & $i$  & $j$  & $k$  \\
		\hline
		1   & 1   & $i$  & $j$  & $k$  \\
		$i$ & $i$ & 1    & $-k$ & $j$  \\
		$j$ & $j$ & $k$  & 1    & $-i$ \\
		$k$ & $k$ & $-j$ & $i$  & 1    \\
	\end{tabular}
\end{center}
every element in $Q_{8}$ has order 2, so every element is its own inverse. We now want to show that every subgroup of $Q_{8}$ is normal.

\begin{example}{Show every subgroup of $Q_{8}$ is normal.}
	By the theorem of Lagrange, we know that the possible orders of subgroups are 1, 2, 4, and 8. The simplest normal subgroup is the center,
	\[
		Z(Q_{8})=\{\pm 1\},
	\]
	which happens to be the only subgroup of order 2.
	We know this is normal by the theorem, so we can consider the other subgroups. For $Q_{8}$, the rest of the subgroups are all generated by either $i$, $j$, or $k$ and have order 4. Consider $<i>=\{\pm 1, \pm i\}$. We need $xix^{-1}\in<i>\ \forall x\in Q_{8}$.
	\begin{gather*}
		x=j\implies jij^{-1}=jij=jk=i\in<i>\\
		x=k\implies kik^{-1}=kik=jk=i\in<i>\\
		\therefore <i>\normal\ Q_{8}.
	\end{gather*}
	We leave it to you to prove that the other generators are normal. Note that the order 8 subgroup of $Q_{8}$ is $Q_{8}$ itself, but it follows immediately from the definition of normality that $G\normal G$. Similarly, the order 1 subgroup of $Q_{8}$ is the trivial subgroup, and it again follows immediately that $\{e\}\normal G$.
\end{example}
The idea of normality will be very important when we talk about factor groups.

\subsection*{Cosets}
Cosets are an expansion on the idea of a group. We have set operations that we can perform on sets that we typically define in an intuitive way; that being said, when we see something like $n\ZZ$, we expect it to act like
\[
	n\ZZ = \{nz:z\in\ZZ\}.
\]
This is actually what we call a coset of $\ZZ$.
\begin{definition}{Coset}
	Let $G$ be a group and let $a\in G$. Then the left coset of $G$ defined by $a$ is the set
	\[
		aG=\{ag:g\in G\}.
	\]
\end{definition}

Note that we call $aG$ the left coset; we could define the right coset in a similar way, namely $Ga$, but the convention is to use left multiplication. Let's take a look at how to find cosets.

\begin{example}{Let $H=\{\pm 1, \pm i\}\leq Q{8}$. Find all the cosets of $H$}
	We only need to consider the positive elements of $Q_{8}$ here since we can simply associate the negatives with each other to cancel them out. Hence,
	\begin{gather*}
		1H=\{\pm 1, \pm i\}\\
		iH=\{\pm i, \pm 1\}\\
		jH=\{\pm j, \pm k\}\\
		kH=\{\pm k, \pm j\}
	\end{gather*}
	These are all the cosets of $H$, so we are done.
\end{example}

The last example points out an interesting fact about cosets. If we have $h\in H$, then $hH=H$, so we typically are interested in cosets defined by elements not in $H$. It follows from this fact that if we have two cosets, then
\[
	aH=bH\text{ or } aH\cap bH=\emptyset.
\]
In other words, they are the same coset or they are completely disjoint. Furthermore, we can conclude that
\[
	a,b\in H\implies |aH|=|bH|=|H|.
\]
There is a theorem that follows from this fact which gives us an even nicer property of cosets.
\begin{theorem}{}
	Let $H\leq G$ and let $a,b\in G$. Then $|aH|=|bH|$.
\end{theorem}
\begin{proof}
	If $a,b\in H$, then we are done. Assume $a,b\not\in H$. We need a bijection $f:aH\to bH$. Define $f$ by
	\[
		f(ah) = bh.
	\]
	We claim $f$ is injective since
	\[
		f(ah_{1})=f(ah_{2})\implies bh_{1}=bh_{2}\implies h_{1}=h_{2}\implies ah_{1}=ah_{2}.
	\]
	Let $bh\in bH$. Then $f(ah)=bh$, so we conclude $f$ is surjective.
\end{proof}
These properties show why we like cosets so much; the disjointness property is actually powerful enough to help us prove the theorem of Lagrange.

\begin{theorem}{Theorem of Lagrange}
	Let $G$ be a finite group with $H\leq G$. Then $ord(H)\mid ord(G)$.
\end{theorem}
\begin{proof}
	Let $G$ be finite and consider the left cosets of $H\leq G$. Then
	\[
		\exists a_{1},a_{2},\dots,a_{n}\suchthat G=\coprod_{i=1}^{n}a_{i}H.
	\]
	Hence,
	\[
		|G| = \left|\coprod_{i=1}^{n}a_{i}H\right| = \sum_{i=1}^{m}|a_{i}H|.
	\]
	Therefore $|a_{i}H|=|a_{j}H|=|H|$, so
	\[
		\sum_{i=1}^{m}|a_{i}H| = \sum_{i=1}^{m}|H| = m|H| = |G|\implies ord(H)\mid ord(G).
	\]
\end{proof}
The properties of cosets are incredibly powerful; they will become even more powerful when we pair them with the properties of normal subgroups.

\subsection*{Factor Groups}
The motivation behind the next topic comes from the idea of congruence. In $\ZZ$, we have the concept of modular arithmetic which produces equivalence classes consisting of elements that are equivalent modulo a particular integer.
We would like to mimic this behavior with groups; it turns out that this is easier said than done.
We introduced the concepts of normality and cosets together because factor groups only make sense when consider a group and its normal subgroups.

\begin{definition}{Factor Group}
	Let $G$ be a group and let $H\normal G$. The the group $\fgroup{G}{H}$ is the factor or quotient group of $G$ consisting of all the cosets of $H$ for each $g\in G$. In other words,
	\[
		\fgroup{G}{H}=\{gH:g\in G\}.
	\]
\end{definition}
We read the expression $\fgroup{G}{H}$ as \textit{G mod H}; this only forms a group when $H\normal G$.
The definition of a factor group links the ideas of cosets and normality as well as introducing the concept of modding out by a group.
Since the factor group forms a group, we need an operation to go with it.
The intuitive choice here is
\[
	aH * bH = (a*b)H,
\]
where $*$ is the operation on the group.
To see that $\fgroup{G}{H}$ is a group note that it contains the identity coset $eH=H$, if we have $gH\in\fgroup{G}{H} \implies g^{-1}H\in\fgroup{G}{H}$, and the operation is associative.
We also need to be certain that the operation is well defined on the cosets, so we need $aH=bH\land \alpha H=\beta H\implies (ab)H=(\alpha\beta)H$.
As it turns out, we need the condition of $H\normal G$ in order for this to hold.

\begin{theorem}{}
	Let $G$ be a group and let $H\normal G$. Then the operation $(aH)(bH)=(ab)H$ is well defined on the cosets.
\end{theorem}
\begin{proof}
	We want to show $aH=bH\land \alpha H=\beta H\implies (ab)H=(\alpha\beta)H$, so suppose such cosets exist in $\fgroup{G}{H}$.
	Then $a\alpha^{-1}\in H$ and $b\beta^{-1}\in H$ by closure and inverses of subgroups. Consider the quantity
	\begin{align*}
		a(b\beta^{-1})\alpha^{-1} &= a(b\beta^{-1})(a^{-1}a)\alpha^{-1}\\
		&= a(b\beta^{-1})a^{-1}(a\alpha^{-1})\\
		&= aha^{-1}(a\alpha^{-1})\\
		&= h_{1}(a\alpha^{-1})\in H\text{ by normality and closure.}
	\end{align*}
	Hence, the operation is well defined on normal subgroups.
\end{proof}

We now have a well defined group with a well defined operation, so we can apply all the facts we know about groups thus far to the factor group.
One important piece of information we like to know is the order of a group.
For finite groups, this turns out to be a very simple calculation involving the index of a group.
\begin{definition}{Index}
	Let $G$ be a group and let $H\leq G$. Then the index of $H$ in $G$ is the number of distinct right or left cosets of $H$, denoted by $[G:H]$.
\end{definition}
The index is relatively easy to compute for finite groups since it is just the quotient of the two groups.
\begin{example}{Determine $[Sym_{n}:Alt_{n}]$.}
\[
	[Sym_{n}:Alt_{n}]=\frac{ord(Sym_{n})}{ord(Alt_{n})}=\frac{n!}{n!/2}=2
\]
\end{example}
The fact that the index is 2 tells us a very important fact about the subgroup.
\begin{theorem}{}
	Let $G$ be a group and let $H\leq G$. Then
	\[
		[G:H]=2\implies H\normal G.
	\]
\end{theorem}
Based on our previous example, we can consider the factor group $\fgroup{Sym_{n}}{Alt_{n}}$, which has order 2.
If $G$ is infinite, then the index no longer tells us the order of the factor group, so we have to analyze the group in a different way.
\begin{example}{Determine the order of $\fgroup{\ZZ}{123\ZZ}$.}
	Without going into too much detail, we can convince ourselves that if we take $123\ZZ+123\ZZ$, we still have $123\ZZ$. In this way, the factor group acts like $\ZZ_{123}$, which we know to have order 123.
\end{example}

This last example points out the most important idea to take away from this chapter; that is, the groups $\fgroup{\ZZ}{123\ZZ}$ and $\ZZ_{123}$ act the same and are roughly equivalent under their respective operations.
We will discuss this idea in great detail in the next section.
