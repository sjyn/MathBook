\section{Introduction to Rings}

In the previous chapter, we looked in depth at the concept of a group.
Recall that groups are defined with one operation, which we called $*$; this need not alway be the case, however.
Groups are simple in this aspect especially when compared to other structures such as a Ring or a Field.
We will begin by defining a Ring.

\begin{definition}{Ring}
    Let $R$ be a set an let $+$ be the standard operation of addition and $\cdot$ be the standard operation of multiplication. We say $(R,+,\cdot)$ is a ring if
    \begin{enumerate}
        \item $(R,+)$ is an abelian group
        \item multiplication is associative
        \item multiplication distributes over addition.
    \end{enumerate}
\end{definition}

We can see that a ring is a much more complex structure than a group, and in fact it contains groups within it, similar to the idea that all squares are rectangles but not all rectangles are squares.
Like many of the structures in algebra, we have dealt with rings without even knowing it.
The groups $(\RR, +)$ and $(\RR^{*},\cdot)$ form a ring $(\RR, +, \cdot)$ when combined.
Note that the definition does \textit{not} require that $R$ be a group under multiplication; rather, we only need that multiplication is associative and that the distributive law holds.
It is this fact that allows us to construct a ring from one of our favorite groups.

\begin{example}{Show $\ZZ_{n}$ is a ring}
    We already know that $\ZZ_{n}$ is abelian under addition by the definition of the operation. We define multiplication in the context to be
    \[
        \overline{ab}=\overline{a}\overline{b}\equiv ab\mod n.
    \]
    Hence
    \begin{align*}
        (\overline{a}\overline{b})\overline{c} &= \overline{ab}\overline{c}\\
        &= \overline{abc}\\
        &= \overline{a}\overline{bc}\\
        &= \overline{a}(\overline{b}\overline{c}).
    \end{align*}
    The distributive law is trivial to show.
\end{example}

The concept of an identity element was crucial to our study of groups.
The identity was determined by the operation over the set, and it is no different for rings.
Here, we have two operations so we designate two identity elements, typically called 1 (for multiplication), and 0 (for addition).
We were also concerned with the concept of inverse elements, but we must be careful.
The additive inverse is guaranteed to us because of the fact that $(R,+)$ is a group, but we are not guaranteed a multiplicative inverse (although it is not impossible).
