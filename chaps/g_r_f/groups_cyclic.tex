\section{Cyclic Groups}
The group formed by $\ZZ_{n}$ turns out to be a perfect group to introduce the concept of cyclic groups. Cyclic groups are exactly what they sound like; if we keep taking powers of elements, we will eventually hit every element in the group. If the group is additive, then we can repeatedly add until we reach every element in the group. The formal definition is as follows.
\begin{definition}{Cyclic Group}
    A group $G$ is called cyclic if $\exists a\in G$ such that
    \begin{enumerate}
        \item (additive) $G=\{na:n\in\ZZ\}$, or
        \item (multiplicative) $G=\{a^{n}:n\in\ZZ\}$.
    \end{enumerate}
    We use the notation $G=<a>$ where $a$ is called the generator of the group.
\end{definition}
In other words, a group is cyclic if it can be generated by an element within itself. Consider the following example.

\begin{example}{Show $\ZZ_{4}$ is cyclic.}
    We can do this by finding an element in $\ZZ_{4}$ such that $\ZZ_{4}=<a>$. It turns out that 1 does the job, since
    \begin{align*}
        1 &= 1\\
        2 &= 1 + 1\\
        3 &= 1 + 1 + 1\\
        0 &= 1 + 1 + 1 + 1
    \end{align*}
\end{example}

\begin{example}{Show $\ZZ$ is cyclic.}
    $\ZZ$ is an infinite group, so we can't list the elements like we did in $\ZZ_{4}$. However, if we recall the definition of cyclic, we know we just need one element that we can multiply by every $n\in\ZZ$ to get every element in the group. Let's try $<1>$:
    \[
        <1>=\{1*n:n\in\ZZ\}=\ZZ,
    \]
    so 1 does the job. It turns out that $-1$ is also a generator of $\ZZ$, but those are the only two elements that work.
\end{example}

One of the benefits of cyclic groups is the fact that we are guaranteed commutativity. Note that this does not work the other way around; there are abelian groups that are not cyclic.

\begin{theorem}{}
    Let $G=<a>$. Then $G$ is abelian.
\end{theorem}
\begin{proof}
    Let $x,y\in G$. Then $\exists m,n\in\NN\suchthat x=a^{m}$ and $y=a^{n}$. Then
    \[
        xy=a^{m}a^{n}=a^{m+n}=a^{n+m}=a^{n}a^{m}=yx.
    \]
\end{proof}
The ability to commute powers is a key property of abelian groups that we will use extensively in later sections of this chapter.

\subsection*{Order}
Let's begin with the formal definition of order.

\begin{definition}{Order}
    Let $G$ be a group. Then the order of $G$ is the cardinality of the set $G$.\\
    Let $a\in G$. The order of $a$ is defined to be
    \begin{enumerate}
        \item The minimal $n\in\NN\suchthat a^{n}=e$, or
        \item $\infty$ if no such $n$ exists.
    \end{enumerate}
    We use the notation $ord(a)=|a|$ and $ord(G)=|G|$.
\end{definition}

The order of a groups is intuitive; it's the order of an element that we are more interested in.

\begin{theorem}{}
    Let $G$ be a group such that $|G|<\infty$. Then $a\in G\implies |a|<\infty$.
\end{theorem}
\begin{proof}
    Let $a\in G$ and suppose $|G|=n$. Consider the set
    \[
        \{a,a^{2},a^{3},\dots,a^{n},a^{n-1}\}.
    \]
    Then $\exists i,j\in\ZZ\suchthat a^{i}=a^{j}$ by the pigeonhole principle. Without loss of generality, assume $i<j$. Then $e=a^{j-i}$, and $j-i\in\NN$, so we conclude that $|a|<\infty$.
\end{proof}
There are two very important corollaries to this theorem.
\begin{corollary}
    If $|G|<\infty$, then every element in $G$ has the property
    \[
        a^{n}=e
    \]
    for some $n\in\NN$.
\end{corollary}
\begin{proof}
    This follows immediately.
\end{proof}

Out last major theorem about order concerns the order of inverses. Luckily for us, the order of an element and the order of its inverse are the same.

\begin{theorem}{}
    Let $G$ be a group and let $a\in G$. Then $|a|=|a^{-1}|$.
\end{theorem}
\begin{proof}
    Suppose $|a|=n$. Then $a^{n}=e$, so
    \begin{align*}
        a^{-n}a^{n}&=a^{-n}e\\
        e&=a^{-n}\\
        e &= (a^{-1})^{n},
    \end{align*}
    so $|a^{-1}|\leq n$.
    Suppose $m=|a^{-1}|$ where $m\leq n$. Then
    \begin{align*}
        (a^{-1})^{m}&=e\\
        a^{-m}&= e\\
        a^{m}a^{-m}&=a^{m}e\\
        e &= a^{m},
    \end{align*}
    which implies $n\leq m$ since $|a|=n$. Therefore $|a^{-1}|=n$.
\end{proof}

Calculating the order of elements in a group is often not practical to do by hand. Luckily, we can calculate the order of an element in $\ZZ_{n}$ by looking at the following.

\begin{theorem}{}
    Let $a\in\ZZ_{n}$. Then
    \[
        |a|=\frac{n}{\gcd{n,a}}.
    \]
\end{theorem}
We won't provide a formal proof here as much of it relies on ideas from Number Theory. This is an interesting fact to keep in mind though when dealing with $\ZZ_{n}$. There are other ways to calculate order for other groups, but we need more background before we discuss that.
