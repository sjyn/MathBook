\section{Walks, Paths, and Trails}

The topics covered in this section address the ways we can traverse graphs.
In our study, we have the freedom to move from any vertex to any other vertex as long as there is an edge between them.
The strategies we use to move around a graph form the theory of walks, paths, and trails.

\begin{definition}{Walk}
	Suppose $G$ is a graph. We say a finite sequence of vertices $v_{1},v_{2},\dots,v_{k}$ is a walk of length $k$ provided that $v_{i}v_{i+1}\in E(G)$ for all $1\leq i\leq k$. If $v_{1}=v_{k}$, then we say the walk is closed.
\end{definition}

In other words, a sequence of vertices is a walk if we can always move along a single edge to the next vertex in the sequence.
We have several ways to write walks; we can use set notation $\{a,b,c,d\}$ or simply write the vertices in the order they appear, $abcd$, if it is obvious to do so.

\begin{example}{}

	\begin{minipage}{0.3\textwidth}
		\begin{tikzpicture}[node distance=1cm]
			\node(a) {a};
			\node(b) [above right of=a] {b};
			\node(c) [below right of=b] {c};
			\node(d) [below of=c] {d};
			\node(e) [below left of=d] {e};

			\draw(a) -- (b);
			\draw(b) -- (c);
			\draw(c) -- (d);
			\draw(b) -- (d);
			\draw(d) -- (e);
		\end{tikzpicture}
	\end{minipage}
	\begin{minipage}[t]{0.65\textwidth}
        \vspace{-1.2cm}
		In this graph, $abde$ and $abcba$ form walks, but $ade$ does not since there is $ad\notin E(G)$.
        Note that we can repeat vertices if need be.
	\end{minipage}

\end{example}

Walks are simplistic in their construction since we can move anywhere in the graph where there is an edge.
The more interesting idea here is a walk where no vertices are repeated.

\begin{definition}{Path}
    Let $W$ be a walk in a graph $G$. If $W$ contains no repeated vertices, then we call $W$ a path. If we include the edge $v_{k}v_{1}$, provided that it exists, the we call the path closed, or a cycle.
\end{definition}

Paths concern vertices, but there is an analogous idea for edges.

\begin{definition}{Trail}
    If all edges in a walk are distinct, then it is called a trail.
\end{definition}

When talking about walks and paths, we say the length of the walk is the number of edges in the walk.
If the path is closed, then we count the missing edge in the path as part of the length.
With these definitions in mind, we now have enough to build our first theorem.

\begin{theorem}{}
    Let $G$ be a graph and let $A$ be a walk in $G$. If $A$ is not a trail, then $A$ is not a path.
\end{theorem}

We won't prove this here, but note that the contrapositive holds (and is a little more pleasing to read)

\begin{theorem}{}
    If $A$ is a path, then $A$ is a trail.
\end{theorem}

Again, we won't prove this, but it is important to keep in mind. The next theorem, however, is a little more worthwhile to prove.

\begin{theorem}{}
    Suppose $G$ is a graph and let $u,v\in V(G)$ be distinct. Then every $uv$ walk contains a $uv$ path.
\end{theorem}
\begin{proof}
    Let $A$ be a $uv$ walk, and assume $A$ has length 1. Hence, $u$ and $v$ are the only vertices in the walk, so $A$ is a path. Assume that all $uv$ walks of length $l$ contain a $uv$ path.
    We must show that for a $uv$ walk $A$ of length $l+1$ contains a $uv$ path.
    Since $A$ is a walk, we may name the vertices in $A$:
    \[
        A = v_{1}v_{2}\dots v_{k}
    \]
    with $u=v_{1}$ and $v=v_{k}$.
    If $A$ is a $uv$ path then we are done, so assume that $A$ is not a path.
    Then $\exists i\neq j \suchthat v_{i}=v_{j}$.
    Define $B$ to be a walk such that
    \[
        B = v_{1}v_{2}\dots v_{i}v_{j+1}v_{j+2}\dots v_{k}.
    \]
    We have removed at least one edge from $A$ to create $B$, so the length of $B$ is less than or equal to $l$.
    By the inductive hypothesis, we have that $B$ contains a $uv$ path $C$.
    Hence if $C$ is contained in $B$ and $B$ is contained in $A$, then $C$ is contained in $A$, so $A$ contains a $uv$ path.
\end{proof}

This is a very important result since it allows us to find a $uv$ path in any $uv$ walk.
