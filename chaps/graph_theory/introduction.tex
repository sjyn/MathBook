\section{Introduction to Graphs}

The idea of a graph is relatively simple but requires some prior ideas from the study of set theory.
Recall that given a set $A$, we can create what is known as the power set of a, $\mathcal{P}(A)$, consisting of all subsets of $A$.
We can extend this idea a little further but considering subsets of the power set of a given set.

\begin{definition}{Power-$k$ Set}
    Let $A$ be a finite set of order $n$ and let $k\in\NN$.
    The power-$k$ set of $A$, $\mathcal{P}_{k}(A)$, is the set of all subsets of $A$ with cardinality $k$.
\end{definition}

We get some nice relationships between the power-$k$ set and the power set of $A$.
First, note that $\mathcal{P}_{k}(A)\subset\mathcal{P}(A)\ \forall k$; even if $k > n$, we still have a subset of $\mathcal{P}(A)$ since $\mathcal{P}_{k}(A)=\emptyset$ in that case.
The power set can be expressed as a union of power-$k$ sets:
\[
    \mathcal{P}(A) = \bigcup_{k=1}^{n}\mathcal{P}_{k}(A).
\]
There are various other properties that follow, but our main purpose of introducing the power-$k$ set is to assist us in our definition of a graph.

\begin{definition}{Graph}
    Let $V$ be a finite set and let $E\subset\mathcal{P}_{2}(V)$.
    The pair $(V,E)$ is called a graph, with the vertex set $V$ and the edge set $E$.
\end{definition}

This is our main definition to keep in mind throughout this entire chapter.
We may often see pictures of graphs drawn with dots and lines, but it is the pair of sets that truly defines the structure.
We can, and usually do, prefer to draw graphs from the provided sets in order to determine various properties.

\begin{example}{Let $V=\{a,b,c\}$ and $E=\{\{a,b\},\{b,c\}\}$. Draw a representation of this graph.}
    We ought to first determine whether or not $(V,E)$ actually is a graph.
    We require $E$ to have two main properties:

    \begin{enumerate}
        \item Every element of every set within $E$ is an element of $V$.
        \item $E\subset\mathcal{P}_{2}{V}$.
    \end{enumerate}

    Indeed, both these properties hold, so $(V,E)$ is a graph.
    We could draw infinitely many representations of $(V,E)$; here is one such.
    \begin{center}
        \begin{tikzpicture}[node distance=1cm]
            \node(a) {a};
    		\node(b) [below left of=a] {b};
    		\node(c) [below right of=a] {c};

    		\draw(a) -- (b);
    		\draw(b) -- (c);
        \end{tikzpicture}
    \end{center}
    The pairs within the $E$ represent the vertices which share an edge; the values within $V$ represent the vertices themselves.
\end{example}

One key fact to note is that the definition does not require the pairs within $E$ to be ordered; if they are ordered, then we have what is called a directed graph.
We also don't have any repeated edges (if we do, we have a multigraph), and no vertex can share an edge to itself (called a pseudograph)
We will look at directed graphs in depth later, but for now we will consider the non-directed case.

The next few pages will concern a whole slew of definitions.
The terminology here is not always standard, but it is what we will go with for the rest of our discussion on graphs.

\begin{definition}{Order}
    Suppose $G$ is a graph. Then the order of $G$ is $ord(G)=|V(G)|$.
\end{definition}
\begin{definition}{Size}
    Suppose $G$ is a graph. Then the size of $G$ is $size(G)=|E(G)|$.
\end{definition}

These terms are more for convenience. The next few are actually useful.

\begin{definition}{Degree}
    Let $v\in V(G)$. The degree of $v$ in $G$, $deg_{G}(v)$, is the number of edges incident with $v$.
    We often write $deg(v)$ if the context of the graph is obvious.
\end{definition}

The degree of a vertex tells us how many edges come out of the vertex.
As it so happens, we actually know how large this number can be: $ord(G) - 1$.
This makes sense intuitively; a vertex can have no duplicate edges and no edges to itself, so the max number of edges a given vertex can have is an edge to every other vertex in the graph.
If every vertex has this property, we call the graph \textit{complete}.

\begin{definition}{Complete}
    Let $G$ be a graph and let $n=ord(G)$. We call $G$ complete if $\forall\ v\in V(G),\ deg(v)=n-1$. Since these graphs are unique, we say $G=K_{n}$, meaning the complete graph on $n$ vertices.
\end{definition}

The concept of graph equality is something that we discuss further in a little bit.
For now, we are ready for our first theorem.

\begin{theorem}{}
    Suppose $G$ is a graph. Then
    \[
        \sum_{v\in V(G)}deg(v) = 2\ size(G).
    \]
\end{theorem}
\begin{proof}
    We can show this via induction. Suppose $size(G)=0$. Then $2\ size(G)=2\cdot0=0$.
    Since $size(G)=0$, $deg(v)=0\ \forall v\in V(G)$, so the sum over the degrees is also 0.
    Hence, we may assume for all graphs $H$ with $n$ edges,
    \[
        \sum_{v\in V(H)}deg_{H}(v) = 2n.
    \]
    Assume $G$ has $n+1$ edges. We must show
    \[
        \sum_{v\in V(G)}deg(v)=2(n+1).
    \]
    Let $H$ be a graph such that $V(H)=V(G)$ and $E(H)=E(G)\setminus ab$ for some $ab\in E(G)$. Then
    \[
        \sum_{v\in V(H)}deg_{H}(v)=2n\implies\sum_{v\in V(G)}deg_{H}(v)=2n.
    \]
    Since $ab\in E(G),\ \exists v_{1},v_{2}\in V(G)$ which are incident with $ab$.
    Hence $deg_{H}(v_{1})+1=deg_{G}(v_{1})$ and $deg_{H}(v_{2})+1=deg_{G}(v_{2})$.
    Note that for all other vertices in $v\in V(G)$, $deg_{H}(v)=deg_{G}(v)$.
    Then
    \[
        \sum_{v\in V(G)}deg_{H}(v)=-2+\sum_{v\in V(G)}deg_{G}(v).
    \]
    By the inductive hypothesis, we have
    \[
        2n=\left( \sum_{v\in V(G)}deg_{G}(v) \right) - 2\implies
        2(n+1)=\sum_{v\in V(G)}deg_{G}(v).
    \]
\end{proof}

Induction is a very common proof technique that we will encounter throughout or study of graphs; the strategy of creating a new graph with properties of our old graph is also a very useful technique.
