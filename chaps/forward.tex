\section*{Forward}
\noindent

Mathematics is a very involved and very broad subject to study. Most of the mathematics we learn in an academic environment is devoted to problem solving; we are given a theorem and told to prove it, or we are presented with a differential equation and and asked to find $y(x)$. Yet few courses actually attempt to explain the history surrounding mathematics. We know of Newton and Leibniz, the creators of Calculus, and we know of Fermat and his contributions towards the field of Number Theory, but we do not have an understanding of the motivation that drover Fermat, or from where the theory that drives Calculus comes. Knowledge of the history of a subject not only allows us to better understand the subject, but also to appreciate the problem solving of which we are accustomed.

In this book, we will look at a large portion of mathematics and the motivating factors that brought about some of the biggest mathematical discoveries of all time. We will also look at certain influential problems and their solutions, since often times the process by which we obtain a solution tells us more than the solution itself. The emphasis of this text will be on mathematics, but it is important to note that many mathematical discoveries were (and still are) driven by the sciences. We will focus more on theory than application.
