\section{Set Theory}
Up to now, we have been skating around the idea of what is formally known as a set. We talked about using logical quantifiers, but $\forall$ and $\exists$ only make sense when we have a collection to talk about. Such a collection is called a

\begin{definition}{\deft{Set}}
	A set is a collection of objects called elements.
\end{definition}

This is perhaps one of the simplest definitions that we will ever see; at the same time, it is also one of the most fundamental ideas in mathematics. We can create sets of anything: numbers, functions, and even other sets. They can be finite or infinite, countable or uncountable, etc. Typically, we write sets using \textit{set notation}:
\begin{gather*}
	A = \{a,b,c,d,\dots\},\text{ read as \textit{A is the set containing $a,b,c,d,\dots$}}\\
	A = \{x : P(x)\},\text{ read as \textit{A is the set of $x$ such that $P(x)$ is true}}.
\end{gather*}
Sometimes, we will see a $|$ in place of $:$.

The purpose of this section is to introduce the concept of sets as well as the operations we can perform on sets themselves. Much of the material will parallel that which we've already seen when talking about logic. Let's look at some special sets.

\subsection*{Special Sets}

\begin{definition}{\deft{Empty Set}}
	The empty set is the set with no elements, written as $\emptyset=\{\}$. The empty set is a subset of every set.
\end{definition}
\begin{definition}{\deft{Universal Set}}
	The universal set is the set of which all sets are a subset, typically written as $\mathcal{U}$.
\end{definition}
The universal set leads us into our next concept. It is common in mathematics to define some sort of structure, like a set, and then talk about a substructure, in this case called a subset.
\begin{definition}{\deft{Subset}}
	Let $A$ be a set. Then $B$ is a subset of $A$, written $B\subset A$, if and only if $x\in B\implies x\in A\ \forall x\in B$. Every set is a subset of itself.
\end{definition}
When we want to show something is a subset, we must show that it satisfies this property. We can perform a simple proof using truth tables of this.

\begin{example}{\egft{Show that $\{1,2,3\}\subset\{-1,0,1,2,3\}$.}}
	Let $P(x)$ be the statement \textit{$x$ is in $\{1,2,3\}$} and $Q(x)$ be the statement \textit{$x$ is in $\{-1,0,1,2,3\}$}. We need to show that $P(x)\implies Q(x)$ for each x in $\{1,2,3\}$. Let's layout a truth table.
	\begin{center}
		\begin{tabular}{c | c | c | c}
			$x$ & $P(x)$ & $Q(x)$ & $P(x)\implies Q(x)$ \\
			\hline
			1   & $T$    & $T$    & $T$            \\
			2   & $T$    & $T$    & $T$            \\
			3   & $T$    & $T$    & $T$
		\end{tabular}
	\end{center}
	Therefore, we conclude $\{1,2,3\}\subset\{-1,0,1,2,3\}$.
\end{example}
That was a very simple example since the sets were small and finite, but what if we have an infinite set? Here are some very common sets that we will see repeatedly throughout this book.
\begin{align*}
	\text{Natural Numbers: } & \NN = \{1,2,3,4,\dots\}                    \\
	\text{Integers: }        & \ZZ = \{\dots,-3,-2,-1,0,1,2,3,\dots\}     \\
	\text{Rationals: }       & \QQ = \left\{\frac{a}{b}:a,b\in\ZZ\right\} \\
	\text{Real Numbers: }    & \RR                                        \\
	\text{Complex Numbers: } & \CC = \{a+bi : a,b\in\RR\}
\end{align*}
Note that we did not define the reals. The definition requires a lengthy discussion that cannot be boiled down to a simple set definition. Simply put, it is all of $\QQ$ and everything else that is not in $\QQ$ up to $\CC$; so numbers like $\pi$ and $e$ are in $\RR$. The complex numbers are also a mysterious set that require a lot of time to understand in great detail. These sets have a special relationship, namely:
\[
	\NN\subset\ZZ\subset\QQ\subset\RR\subset\CC.
\]
This list is certainly not comprehensive either. There are sets that exist above $\CC$ and sets that exist between $\QQ$ and $\RR$.

Recall that we said we could have sets containing sets. The canonical example of this phenomenon is know as the Power Set.
\begin{definition}{\deft{Power Set}}
	Let $X$ be a set. Then the power set of $X$, written as $\mathcal{P}(X)$, is the set of all subsets of $X$.
\end{definition}
The power set is a useful tool that we will look use later on. For now, consider the following example.
\begin{example}{\egft{Find the Power Set of $\{1,2,3\}$.}}
	For simplicity, let's call $X$ the set $\{1,2,3\}$. We are being asked to find $\mathcal{P}(X)$. To do this, we must list off all the subsets of $X$.
	\[
		\begin{array}{l l l l}
			\{1\}   & \{1,2\} & \{1,2,3\} & \{2\} \\
			\{2,3\} & \{3\}   & \{1,3\}   &
		\end{array}
	\]
    The trick here is to notice that there is one more subset that we haven't listed yet: $\emptyset$. Therefore,
    \[
        \mathcal{P}(X) = \{\{1\}, \{1,2\}, \{1,2,3\}, \{2\},\{2,3\}, \{3\}, \{1,3\}, \emptyset\}.
    \]
\end{example}
Finding the power set can be rather tedious, but remember that $\emptyset$ is in the power set.

\subsection*{Operations on Sets}
We can perform basic operations on numbers, and we would like to mimic this behavior with sets. We need to ensure that the operation is \textit{well defined}. That is, the operation makes sense for any two sets.
