\section{Introduction to Proofs}

\begin{namedtheorem}[Russell's Paradox]
    Let $R=\{X : X\notin X\}$. Then $R\in R\iff R\notin R$. In other words, we let $R$ be the set of all sets that do not contain themselves. Then $R$ contains itself if and only if $R$ does not contain itself.
\end{namedtheorem}
Based on our knowledge of the biconditional implication, we know that this is impossible. This was a problem that plagued mathematicians for years; it seemed to break the basic assumptions of set theory, which would in turn break the majority of mathematics. The solution to this paradox involves very advanced material, so we won't look at it here. The reason we mention it is to point out an idea that permeates mathematics: we have to, at some point, make assumptions without proof. When we say mathematics is built on proof, we mean that we start with fundamental assumptions and move on from there.
