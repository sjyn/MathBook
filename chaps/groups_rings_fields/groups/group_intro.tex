\section{Introduction}
Our study of algebraic structures begin with the idea of a group.
As it turns out, we have actually seen examples of groups without even knowing it: $\ZZ$, $\RR$, $\CC$, and $\ZZ_{p}$ are just a few groups.
Groups are defined chiefly by a set as well as what we call a \textit{binary operator}.
\begin{definition}{\deft{Binary Operation}}
	Let $S$ be a set. A binary operation on $S$ is a mapping
	\[*:S\times S \to S.\]
	It is a means of associating a unique third object to each pair of objects in $S$.
\end{definition}
We have also seen various binary operations such as addition and multiplication. Note that the definition of binary operation implies that the operation is closed; that is, the elements must land back in $S$ after applying the operation. Hence, operations like subtraction are not always binary operations.
\begin{example}{\egft{Show that subtraction is a binary operation on $\ZZ$, but not on $\NN$}}
    To see this, let $a,b\in\ZZ$. It is clear that $a-b\in\ZZ\ \forall a,b\in\ZZ$. However, take $a,b\in\NN\suchthat\ a<b$. Then $a-b<0\implies a-b\notin \NN$.
\end{example}
There are other binary operations that are less common, such as $a*b=\max{(a,b)}$ on $\RR$, or $f\circ g$ where $f$ and $g$ are functions. We want to keep these operations in mind, as they will play an important role in later sections.
The operator is what determines the properties which we require for a group; we will define a group by introducing substructures of a group that together form a group.
The first structure is called a \textit{semigroup}, and is defined as
\begin{definition}{Semigroup}
	Let $S$ be a set and $*$ be a binary operation. A semigroup is the pair $(S,*)$ such that
	\[
		a,b,c\in S\implies (a*b)*c=a*(b*c).
	\]
\end{definition}
In other words, the semigroup is a set paired with an associative operation.
An example of a semigroup would be $(\ZZ,+)$, or the integers under addition; we know that the operation of addition is associative on the integers.
From a semigroup we can construct what is known as a monoid.
\begin{definition}{Monoid}
	Let $(M,*)$ be a semigroup. Then $M$ is a monoid if
	\[
		\exists e\in M\suchthat m*e=e*m=m\ \forall m\in M.
	\]
\end{definition}
Hence we introduce the concept of an identity element typically called $e$.
This identity element, when combined with the binary operation, should map every element back to itself; it is no surprise that $(\ZZ,+)$ is also a monoid, since $0+a=a+0=a\ \forall a\in\ZZ$.
A group turns out to be nothing more than a monoid with one extra property.

\begin{definition}{\deft{Group}}
	Let $G$ be a set and $*$ be a binary operator. Then the pair $(G,*)$ is a group provided that
	\begin{enumerate}
		\item $*$ is associative; that is $(a*b)*c=a*(b*c)$.
		\item There is an identity element $e\in G$; that is $a*e=e*a=a \forall a\in G$.
		\item Every element in $G$ has an inverse; that is $a\in G\implies a^{-1}\in G \suchthat a*a^{-1}=a^{-1}*a=e\ \forall a\in G$.
	\end{enumerate}
\end{definition}
This is the key definition of this chapter.
When we want to prove something is a group, we must prove all three axioms, typically called the group axioms, listed in the definition.
Often associativity is the most difficult; as such, it is often helpful to try and prove identity first followed by inverses.
Note that the group is a set \textit{and} an operator.
It is the operator that determines what the identity is, what the inverse of an element is, and whether or not elements associate.
Let's look at an example of proving something is a group.

\begin{example}{\egft{Show that $(\ZZ,+)$ is a group.}}
    We first need an identity element. The identity will satisfy the property that $a+e=a\ \forall a\in\ZZ$. It is clear that 0 is the element we need.
    Next, we need to ensure that inverses exist for every element in $\ZZ$. The additive inverse is simply the negation of the element: $a+-a=0$. We know that $a\in\ZZ\implies -a\in\ZZ$, so we have inverses.
    Finally, we need associativity to hold. Luckily, addition is an associative operator, so
    \[
        a,b,c\in\ZZ\implies (a+b)+c=a+(b+c).
    \]
\end{example}

The integers under addition are an example of a group that we've seen before.
In fact $(\RR,+), (\CC,+)$, and $(\QQ,+)$ all form groups under addition, but we can certainly form groups of different objects with different operations.
Consider the power set of a set.
If we pair the power set with the symmetric difference operation then we actually get a group, as the following example shows.

\begin{example}{\egft{Let $X$ be a set, and let $\mathcal{P}(X)$ be the power set of $X$. Show $(\mathcal{P}(X),\Delta)$ is a group.}}
    Recall that $\Delta$ is the symmetric difference operator, defined as $A\Delta B= A\cup B\setminus A\cap B$. We won't do the associativity here (see problem \ref{prob:sym_dif_ass}), but it turns out that it is. To show the identity, we need a set $E\subset X \suchthat A\Delta E=A\ \forall A\in\mathcal{P}(X)$. It turns out that the emptyset does the job:
    \begin{align*}
        A\Delta \emptyset &= A\cup\emptyset \setminus A\cap\emptyset\\
        &= A \setminus \emptyset\\
        &= A.
    \end{align*}
    For inverses, we need $A'\suchthat A'\Delta A=\emptyset\ \forall A\in\mathcal{P}(X)$. The answer here is $A'=A$.
    \begin{align*}
        A \Delta A &= A \cup A \setminus A\cap A\\
        &= A \setminus A\\
        &= \emptyset.
    \end{align*}
\end{example}

Up to now, we've shown the existence of inverses and identities within groups, but what about uniqueness?
Luckily, the binary operations do in fact guarantee uniqueness for various reasons that we will see later.
We can prove the uniqueness properties with relative ease.

\begin{theorem}{}
	Let $*$ be a binary operation on a set $G$. If $G$ contains an identity element $e$, then $e$ is unique.
\end{theorem}
\begin{proof}
	The typical way we show uniqueness is to suppose two elements are different and show they are the same. Let $e$ and $f$ be identities in $G$. Then
	\[
		e*f=e=f=e*f,
	\]
	so $e=f$.
\end{proof}
\begin{theorem}{}\label{thm:inv_unique}
	Let $*$ be a binary operation on a set $G$. Then if $G$ is a group, the inverse of an element $a\in G$ is unique.
\end{theorem}
\begin{proof}
	Let $a'$ and $a''$ be inverses of $a$. Then $a*a'=e=a*a''$. Hence
	\begin{align*}
		a*a' &= a*a''\\
		a'*a*a' &= a'*a*a''\\
		e*a' &= e*a''\\
		a' &= a''
	\end{align*}
\end{proof}

One important idea to take away from the last proof is that we are never guaranteed commutativity of elements.
We chose to perform the operation on the left on one side of the equality, so we \textit{must} also perform the operation on the left on the other side of the equation.
When we do have commutativity of elements, we get very special properties; hence we give such groups a special names.
Groups in which every elements commutes are called \textit{abelian groups}.
\begin{definition}{Abelian Group}
	Let $(G,*)$ be a group. If $a*b=b*a\ \forall a,b\in G$, then we call $G$ abelian.
\end{definition}
This will turn out to be an incredibly important distinction as we talk more about groups as abelian groups tend to be much nicer to deal with than non-abelian groups.

\subsection*{Cayley Tables}
Recall that we mentioned $(\ZZ,+)$ is a group; in fact it is an infinite abelian group, but this not need always be true.
For finite groups (i.e. groups in which the defining set has finite size), we can create what is known as a Cayley (or multiplication) table to show the relationships between elements.
Cayley tables, named after mathematician Arthur Cayley, can be used to compare groups, show that the group is abelian, and much more.
Let's look at an example of how to construct such a table.

\begin{example}{Construct the Cayley table for the $V_{4}$ group.}
	Before we begin, it would help to know what the $V_{4}$ group is.
	The name comes from the German word \textit{Vierergruppe} which translates to \textit{four-group}.
	You may have guessed that is has 4 elements, which we call $a,b,c,$ and $e$ where $e$ is the identity.
	The Cayley table will help us to see how the operation is defined.
	\[
		\begin{array}{c | c c c c}
			& e & a & b & c\\
			\hline
			e & e & a & b & c\\
			a & a & e & c & b\\
			b & b & c & e & a\\
			c & c & b & a & e
		\end{array}
	\]
	We can read the operation via $row * column$, i.e. $a * b=c$.
\end{example}

We can use Cayley tables to compare groups as well as show how the binary operation on the group is performed.
Note that this is not a proof that two groups are the same; we will discuss in depth what it means for two groups to be the same later.
Let us look at another example of creating a Cayley table.

\begin{example}{Let $G=\ZZ_{4}$. Construct the Cayley table for $(G,+)$}
	We have seen the group $\ZZ_{4}$ before; it is the integers modulo 4. As is turns out, the Cayley table looks very similar to that of $V_{4}$.
	\[
		\begin{array}{c | c c c c}
			& 0 & 1 & 2 & 3\\
			\hline
			0 & 0 & 1 & 2 & 3\\
			1 & 1 & 0 & 3 & 2\\
			2 & 2 & 3 & 0 & 1\\
			4 & 3 & 2 & 1 & 0
		\end{array}
	\]
\end{example}

From our discussion of Number Theory, we know that the set $\ZZ_{n}$ is very important; it turns out that $\ZZ_{n}$ forms a group.

\subsection*{The Integers Modulo $n$}
When we talked about Cayley tables, we mentioned the group $\ZZ_{n}$ under addition. As it turns out, $\ZZ_{n}$ is a group for every $n\in\NN$ under addition. The addition in $\ZZ_{n}$ is not technically the same addition that we have in $\ZZ$, and the element $1\in\ZZ_{n}\neq 1\in\ZZ$, strictly speaking. Rather, the elements are \textit{equivalence classes}. We talked about these in the chapter on Number Theory and we showed that the operation was well defined.
We now need to show that $\ZZ_{n}$ forms a group.
\begin{theorem}{}
	The integers modulo $n$, known as $\ZZ_{n}$ form a group under the defined operation on equivalence classes called $+$.
\end{theorem}
\begin{proof}
	We will start by showing that there is an identity element in $\ZZ_{n}$. Intuitively, this is the equivalence class of 0, $\overline{0}$.
	\[
		\overline{a}+\overline{0}=\overline{a+0}=\overline{a}.
	\]
	Next, we need inverses. Let $\overline{a}\in\ZZ_{n}$. Then
	\[
		\overline{-a}\equiv\overline{n-a}\implies \overline{a}+\overline{n-a}=\overline{a+n-a}=\overline{n}=\overline{0}.
	\]
	Finally, we require associativity.
	\[
		\overline{a}+(\overline{b}+\overline{c})=\overline{a}+(\overline{b+c})
		=\overline{a+(b+c)}=\overline{(a+b)+c}=\overline{(a+b)}+\overline{c}=
		(\overline{a}+\overline{b})+\overline{c}.
	\]
\end{proof}
We will drop the bar symbol from here on out since the operation is equivalent to addition.
We have shown that $\ZZ_{n}$ is a group under addition, but what about multiplication? In Number Theory, we saw that we were only guaranteed multiplicative inverses when $n$ was prime. It follows that $(\ZZ_{p},\cdot)$ is a group if and only if $p$ is prime. This is left to you to prove, but it is similar to the proof for addition.

% \begin{theorem}{}
% 	Let $n\in\NN$ and $\overline{a}$ define the equivalence class of $a$ modulo $n$. We define the operation on $\ZZ_{n}$ by
% 	\begin{enumerate}
% 		\item $\overline{a}+\overline{b}=\overline{a+b}$ and
% 		\item $\overline{a}\overline{b}=\overline{ab}$.
% 	\end{enumerate}
% 	This operation is well defined.
% \end{theorem}
% \begin{proof}
% 	We will start by proving $\overline{a}+\overline{b}=\overline{a+b}$ is well defined. Consider $\overline{a}=\overline{\alpha}$.
% \end{proof}
